%        File: Documentation.tex
%     Created: Wed Feb 03 11:00 AM 2021 S
% Last Change: Wed Feb 03 11:00 AM 2021 S
%
\documentclass[a4paper]{report}
% \usepackage[hidelinks]{hyperref}

\usepackage{spverbatim}

\usepackage{setspace}
\usepackage{listings}
\usepackage{color}
\definecolor{lightgray}{rgb}{.9,.9,.9}
\definecolor{darkgray}{rgb}{.4,.4,.4}
\definecolor{purple}{rgb}{0.65, 0.12, 0.82}

\lstdefinelanguage{JavaScript}{
  keywords={typeof, new, true, false, catch, function, return, null, catch, switch, var, if, in, while, do, else, case, break},
  keywordstyle=\color{blue}\bfseries,
  ndkeywords={class, export, boolean, throw, implements, import, this},
  ndkeywordstyle=\color{darkgray}\bfseries,
  identifierstyle=\color{black},
  sensitive=false,
  comment=[l]{//},
  morecomment=[s]{/*}{*/},
  commentstyle=\color{purple}\ttfamily,
  stringstyle=\color{black}\ttfamily,
  morestring=[b]',
  morestring=[b]"
}

\lstset{
   language=JavaScript,
   backgroundcolor=\color{lightgray},
   extendedchars=true,
   basicstyle=\footnotesize\ttfamily,
   showstringspaces=false,
   showspaces=false,
   numbers=left,
   numberstyle=\footnotesize,
   numbersep=9pt,
   tabsize=2,
   breaklines=true,
   showtabs=false,
   captionpos=b
}

% \usepackage{blindtext}
\title{Documentation of Back-end Application for Unilabo Website}
\author{Scriptpainter}

\begin{document}
\maketitle
% \linespread{1.5}
\onehalfspacing

\tableofcontents
\newpage

\chapter{Endpoints}

Aplikasi back-end untuk halaman web aplikasi Unilabo dikembangkan dalam bentuk API (\textit{Application Programming Interface}) dengan menggunakan teknologi Node.js dan framework Express.js serta menggunakan MongoDB sebagai \textit{database}. Daftar API dapat dilihat secara detail sebagai berikut.

\section{Focal Point}
\subsection{Post Focal Point}
Interface ini digunakan untuk melakukan \textit{post} deskripsi dan \textit{upload} gambar untuk \textit{Focal Point}. Hasil upload untuk sementara disimpan dalam folder \spverb|temp_uploadedfiles|.
\begin{itemize}
  \item URL: \verb|localhost/api/v1/focal-point/post| 
  \item Method: \textbf{POST} 
  \item \textbf{Input}: Form-Data; description (string), \verb|url_banner| (string)
    \begin{lstlisting}
{
  "description": "Test Post",
  "url_banner": "RobloxScreenShot20210103_192746586.png"
}
    \end{lstlisting}
  \item \textbf{Result}:
    \begin{lstlisting}
{
  "_id": "601ab6044926f93198d6f830",
  "description": "Test Post",
  "url_banner": "temp_uploadedfiles\\1612363268277-RobloxScreenShot20210103_192746586.png",
  "createdAt": "2021-02-03T14:41:08.329Z",
  "updatedAt": "2021-02-03T14:41:08.329Z",
  "__v": 0
}
    \end{lstlisting}

\end{itemize}
                    
\subsection{Update Focal Point (Single) }
Interface berikut ini digunakan untuk melakukan update deskripsi dan/atau gambar untuk \textit{Focal Point}. File yang di-upload untuk sementara disimpan ke dalam folder sementara \verb|temp_uploadedfiles|.
\begin{itemize}
  \item \textbf{URL}: \verb|localhost/api/v1/focal-point/update|
  \item \textbf{Method}: \textbf{POST} 
  \item \textbf{Input}: Form-Data - \verb|_id| (string), description (string, optional), \verb|url_banner| (file.jpg/file.png)
    \begin{lstlisting}
{
  "_id": "601ab6044926f93198d6f830",
  "description": "test update",
  "url_banner": "robloxScreenShot20210103_152510171.png"
}
    \end{lstlisting}
  \item \textbf{Result}:
    \begin{lstlisting}
{
  "_id": "601ab6044926f93198d6f830",
  "description": "Test Update",
  "url_banner": "temp_uploadedfiles\\1612363686974-RobloxScreenShot20210103_152510171.png",
  "createdAt": "2021-02-03T14:41:08.329Z",
  "updatedAt": "2021-02-03T14:48:07.008Z",
  "__v": 0
}
    \end{lstlisting}
\end{itemize}

\subsection{Get One Focal Point}
Interface ini digunakan untuk mengambil satu data \textit{Focal Point} menggunakan parameter \verb|id| yang diambil dari \textit{property} \verb|_id|.
\begin{itemize}
  \item \textbf{URL}: \verb|localhost/api/v1/focal-point/get/:id| 
  \item \textbf{Method}: POST
  \item \textbf{Input}: Parameter - id (string)
    \begin{lstlisting}
localhost:3000/api/v1/focal-point/get/601ab6044926f93198d6f830
    \end{lstlisting}
  \item \textbf{Result}:
    \begin{lstlisting}
{
  "_id": "601ab6044926f93198d6f830",
  "description": "Test Update",
  "url_banner": "temp_uploadedfiles\\1612363686974-RobloxScreenShot20210103_152510171.png",
  "createdAt": "2021-02-03T14:41:08.329Z",
  "updatedAt": "2021-02-03T14:48:07.008Z",
  "__v": 0
}
    \end{lstlisting}
\end{itemize}
\section{Thank You Page}
 
\subsection{Load "Thank You" Page}
Interface ini digunakan untuk membuka halaman \textit{Thank You} yang dilakukan oleh \textit{external user/customer} dengan menggunakan URL yang telah di-generate oleh \textit{internal user}.
\begin{itemize}
  \item \textbf{URL}: \verb|localhost/api/v1/thank-you/:id| 
  \item \textbf{Method}: GET
  \item \textbf{Input}: URL Parameter - \verb|id|.
    \begin{lstlisting}
{
  localhost:3000/api/v1/thank-you/601aec6d95e43d652cd46177
}
    \end{lstlisting}
  \item \textbf{Result}:
    \begin{lstlisting}
{
  "is_submitted": 0,
  "_id": "601aec6d95e43d652cd46177",
  "email": "user@mail.com",
  "createdAt": "2021-02-03T18:33:17.616Z",
  "updatedAt": "2021-02-03T18:35:48.254Z",
  "__v": 0
}
    \end{lstlisting}
\end{itemize}

\subsection{User Submission}
Interface ini digunakan untuk melakukan proses \textit{submit} oleh \textit{external user/consumer} pada halaman \textit{Thank You}.
\begin{itemize}
  \item \textbf{URL}: \verb|localhost/api/v1/thank-you/submit| 
  \item \textbf{Method}: POST
  \item \textbf{Input}: JSON - \verb|_id| (string), email (string), \verb|phone_number| (string)
    \begin{lstlisting}
{
  "_id": "601aec6d95e43d652cd46177",
  "email": "user@mail.com",
  "phone_number" : "081212349876"
}
    \end{lstlisting}
  \item \textbf{Result}:
    \begin{lstlisting}
{
  "_id": "601aec6d95e43d652cd46177",
  "email": "user@mail.com",
  "phone_number" : "081212349876"
}
    \end{lstlisting}
\end{itemize}
         
\subsection{Get All Submission}
Interface ini berfungsi untuk mengambil semua data \textit{user submission}.
\begin{itemize}
  \item \textbf{URL}: \verb|localhost/api/v1/thank-you/get-all|
  \item \textbf{Method}: GET
  \item \textbf{Input}: -
  \item \textbf{Result}:
    \begin{lstlisting}
[
  {
    "is_submitted": 1,
    "_id": "601aec6d95e43d652cd46177",
    "email": "user@mail.com",
    "createdAt": "2021-02-03T18:33:17.616Z",
    "updatedAt": "2021-02-03T18:35:48.254Z",
    "__v": 0
  }
]
    \end{lstlisting}
\end{itemize}
\subsection{Generate URL (Single)}
          Interface ini berfungsi untuk melakukan \textit{generate} URL halaman Thank You untuk satu user. \\ \\
          URL: \verb|localhost/api/v1/thank-you/generate-url| \\
          Method: POST
\subsection{Generate URL (Multiple)}


\section{User Management}
      % user management
\subsection{User Registration}
          Interface ini berfungsi untuk mendaftarkan user dengan mengisi nama, email, dan password. \\ \\
          URL: \verb|localhost/api/v1/user/register| \\
          Method: POST \\
          \textbf{Input}: JSON - name (string), email (string), password (string)
            \begin{lstlisting}
{
    "name": "Administrator",
    "email": "admin@unilabo.com",
    "password": "password"
}
            \end{lstlisting}
                    \textbf{Result}: 
            \begin{lstlisting}[language=TeX]
{
  "data": {
      "_id": "601a57910919825394e747a8",
      "name": "Administrator",
      "email": "admin@unilabo.com",
      "password": "$2a$09$2Jj0yxpUofPoKeZWvBBKweoF6JOm4ImlBXGv8zq54T7WvwNueiTyy",
      "tokens": [],
      "createdAt": "2021-02-03T07:58:09.992Z",
      "updatedAt": "2021-02-03T07:58:09.992Z",
      "__v": 0
  }
}
            \end{lstlisting}
            \subsection{User Login}
          \textbf{Input}: JSON - email (string), password (string)
          \begin{lstlisting}[language=TeX]
{
  "email": "admin@unilabo.com",
  "password": "password"
}
          \end{lstlisting}
\end{document}


